% Options for packages loaded elsewhere
\PassOptionsToPackage{unicode}{hyperref}
\PassOptionsToPackage{hyphens}{url}
%
\documentclass[
  11pt,
  a4paper,
]{article}
\usepackage{amsmath,amssymb}
\usepackage{lmodern}
\usepackage{iftex}
\ifPDFTeX
  \usepackage[T1]{fontenc}
  \usepackage[utf8]{inputenc}
  \usepackage{textcomp} % provide euro and other symbols
\else % if luatex or xetex
  \usepackage{unicode-math}
  \defaultfontfeatures{Scale=MatchLowercase}
  \defaultfontfeatures[\rmfamily]{Ligatures=TeX,Scale=1}
  \setmainfont[]{TeX Gyre Heros}
\fi
% Use upquote if available, for straight quotes in verbatim environments
\IfFileExists{upquote.sty}{\usepackage{upquote}}{}
\IfFileExists{microtype.sty}{% use microtype if available
  \usepackage[]{microtype}
  \UseMicrotypeSet[protrusion]{basicmath} % disable protrusion for tt fonts
}{}
\makeatletter
\@ifundefined{KOMAClassName}{% if non-KOMA class
  \IfFileExists{parskip.sty}{%
    \usepackage{parskip}
  }{% else
    \setlength{\parindent}{0pt}
    \setlength{\parskip}{6pt plus 2pt minus 1pt}}
}{% if KOMA class
  \KOMAoptions{parskip=half}}
\makeatother
\usepackage{xcolor}
\usepackage[margin=0.6in]{geometry}
\setlength{\emergencystretch}{3em} % prevent overfull lines
\providecommand{\tightlist}{%
  \setlength{\itemsep}{0pt}\setlength{\parskip}{0pt}}
\setcounter{secnumdepth}{-\maxdimen} % remove section numbering
\pagestyle{empty}
\usepackage{enumitem}
\usepackage{multicol}

\setlist[description]{style=multiline, leftmargin=40mm, labelsep=5mm}
\setlist[itemize]{noitemsep}
\setlength{\multicolsep}{\parskip}

\newcommand{\ruledheader}[2]{%
\begingroup
\setlength{\fboxsep}{0pt}%
\colorbox{#1}{%
\parbox[b][1.2ex][t]{35mm}{\begin{tiny}\ \end{tiny}}}%
\parbox[b][1.2ex][t]{5mm}{\begin{tiny}\ \end{tiny}}%
\uppercase{\textbf{#2}}
\endgroup}
\ifLuaTeX
  \usepackage{selnolig}  % disable illegal ligatures
\fi
\IfFileExists{bookmark.sty}{\usepackage{bookmark}}{\usepackage{hyperref}}
\IfFileExists{xurl.sty}{\usepackage{xurl}}{} % add URL line breaks if available
\urlstyle{same} % disable monospaced font for URLs
\hypersetup{
  hidelinks,
  pdfcreator={LaTeX via pandoc}}

\author{}
\date{}

\newcommand{\name}{Pascal Lasnier}

\newcommand{\email}{py@lasnier.com}
\newcommand{\phone}{+44 7521 986848}
\newcommand{\website}{github.com/pylasnier}
\newcommand{\address}{St.~Catharine's College, Cambridge, CB2 1RL}

\begin{document}

% \begingroup
% \setlength{\fboxsep}{0pt}%
% \colorbox{blue}{%
% \parbox[b][1.2ex][t]{4em}{\begin{tiny}\ \end{tiny}}}%
% \parbox[b][1.2ex][t]{0.7em}{\begin{tiny}\ \end{tiny}}%
% Pascal Lasnier
% \endgroup
% \ruledheader{blue}{Pascal Lasnier}
\ifdefined\name
\begin{Huge}
\hspace{8mm}
\textsc{\textbf{\name}}
\end{Huge}
\fi

\vspace{-7.5mm}

\begin{flushright}
\mbox{\rule[-6mm]{0.8pt}{14mm}\hspace{3mm}%
\parbox{40mm}{\email\\\phone\\\website}}
\end{flushright}

\ifdefined\address
\vspace{-1.2ex}
\hspace{8mm}
\address
\fi

\begin{Large}\vspace{2ex}

\ruledheader{cyan!50!teal}{Education}\end{Large}

\begin{description}
\item[2020 --- present]
\textbf{University of Cambridge, St.~Catharine's College}\\
BA (Hons) and MEng Engineering, 3\textsuperscript{rd} year student

Years 1 and 2: 1\textsuperscript{st} (76\% in Year 2)

Studying for Aerospace and Aerothermal Engineering
\item[2018 --- 2020]
\textbf{Richard Huish College, Taunton} (A-Levels)

\begingroup
\begin{multicols}{3}
\setlist{label=, leftmargin=0mm}

\begin{itemize}
\tightlist
\item
  Mathematics (A*)
\item
  Further Mathematics (A*)
\item
  Computer Science (A*)
\item
\item
  Physics (A*)
\item
\end{itemize}

\end{multicols}
\vspace{-\parskip}\endgroup
\item[2013 --- 2018]
\textbf{Bishop Fox's School, Taunton} (GCSEs)

7 grade 9s (incl.~Mathematics, Physics, Computer Science, and English
Language)
\end{description}

\begin{Large}

\vspace{-1.5ex}\rule{\textwidth}{0.8pt}\vspace{2ex}

\ruledheader{cyan!50!teal}{Programming experience}\end{Large}

\textbf{2\textsuperscript{nd} year Engineering Robot Project} \textbar{}
2021 \textbar{} Arduino C++ \textbar{}
\emph{github.com/pylasnier/idp205}

\begin{itemize}
\tightlist
\item
  Lead software component of six-person team group project to design an
  autonomous robot;
\item
  Task involved navigation within an arena to search and collect small
  dummies;
\item
  Developed an understanding for the limitations of microcontrollers and
  how to work around them, especially in memory;
\item
  Learnt alternatives for debugging a microcontroller system when
  breakpoints, watches, and other debugging features are not available.
\end{itemize}

\textbf{A-Level Computer Science NEA} \textbar{} 2019 --- 2020
\textbar{} C\# \textbar{} \emph{github.com/pylasnier/functional-studio}

\begin{itemize}
\tightlist
\item
  Designed a very simple, strongly-typed, pure functional programming
  language, which included some basic functional programming features:

  \begin{itemize}
  \tightlist
  \item
    functions as first-class citizens,
  \item
    higher-order functions,
  \item
    selection and recursion,
  \item
    a basic type system including integers, floats, and bools (no arrays
    or monads);
  \end{itemize}
\item
  Developed an intermediate representation (IR) that implements this
  language;
\item
  Built a translator, including a tokeniser and a parser that produce
  the described IR, featuring a rich error system including type
  checking;
\item
  Packaged the whole interpreter with a simple IDE built using Windows
  Forms.
\end{itemize}

\end{document}
